\documentclass[../../master.tex]{subfiles}

\graphicspath{{./image/}}

\begin{document}

\setcounter{chapter}{2}
\chapter{2次元の共形場理論}
\section{2次元の共形対称性}
計量テンソルの変換性を通して共形変換を定義した。
その際、共形 Killing 方程式を解いていくと\(d=1\)や\(d=2\)では条件が緩くなることがわかった。
これはつまり低次元では対称性がより大きくなることがわかる。
物性への応用といったときもここら辺が気になる点である。
ここでは\(d=2\)の2次元系を見ていこう。

この章では座標の無限小変換といったときには
\begin{equation*}
    x^\mu \rightarrow x'^\mu = x^\mu + \epsilon^\mu
\end{equation*}
のように符号を取り換える。
つまり場そのものを並進操作したり回転させていく。
2次元の共形変換の無限小パラメータの満たすべき式は
(2.12)式と(2.13)式を組み合わせて、
\begin{equation}
    \partial_\mu \epsilon_\nu + \partial_\nu \epsilon_\mu = \partial_\rho \epsilon^\rho \eta_{\mu\nu}
\end{equation}
となる。今はユークリッド化をしているのを思い出そう。
この式を成分で書き下すと
\begin{align}
    \pdv{\epsilon_1(x_1,\,x_2)}{x^1} &= \pdv{\epsilon_2(x_1,\,x_2)}{x^2} \\
    \pdv{\epsilon_1(x_1,\,x_2)}{x^2} &= -\pdv{\epsilon_2(x_1,\,x_2)}{x^1}
\end{align}
となる。
これは複素関数論における正則関数のコーシーリーマンの方程式になっている。
そこで複素座標を
\begin{equation}
    z = x^1 + ix^2,\quad \bar{z} = x^1 -ix^2
\end{equation}
で定義する。
これは\(x_1\)を空間座標、\(x_2\)を時間座標と思ってみると光円錐座標に相当する。
また、無限小パラメータの方も
\begin{equation}
    \epsilon \equiv \epsilon^z = \epsilon^1 + i\epsilon^2,\quad \bar{\epsilon} \equiv \epsilon^{\bar{z}} = \epsilon^1 - i\epsilon^2
\end{equation}
というように定義する。
これを導入すると、
\begin{align*}
    \pdv{z}
        &= \pdv{x^1}{z}\pdv{x^1} + \pdv{x^2}{z}\pdv{x^2} \\
        &= \qty(\pdv{z} \frac{z+\bar{z}}{2})\pdv{x^1} + \qty(\pdv{z} \frac{z-\bar{z}}{2i})\pdv{x^2}\\
        &= \frac{1}{2}\qty(\pdv{x^1}-i\pdv{x^2})
\end{align*}
というように微分を書き換えることができるので
(3.2)式と(3.3)式は
\begin{equation}
    \bar{\partial}\epsilon(z,\,\bar{z}) = 0,\quad \bar{\partial}\epsilon(z,\,\bar{z})=0
\end{equation}
のように書き直せる。
\footnote{微分形式でかくこともできそうである。}%: TODO ちゃんと調べる。
ここで微分演算子を
\begin{align}
    \partial \equiv \pdv{z} &= \frac{1}{2}\qty(\pdv{x^1}-i\pdv{x^2})\\
    \bar{\partial} \equiv \pdv{\bar{z}} &= \frac{1}{2}\qty(\pdv{x^1}+i\pdv{x^2})
\end{align}
とおいた。

複素座標を用いると線素は
\begin{equation}
    ds^2 = \eta_{\mu\nu}dx^\mu dx^\nu = dz d\bar{z}
\end{equation}
のように書ける。
これより計量テンソルは
\begin{equation}
    \begin{split}
        &g_{z\bar{z}} = g_{\bar{z}z} = \frac{1}{2},\quad g^{z\bar{z}} = g^{\bar{z} z} = 2\\
        &g_{zz} = g_{\bar{z}\bar{z}} = g^{zz} = g^{\bar{z}\bar{z}} =0
    \end{split}
\end{equation}
とわかる。

ここまでは無限小変換を考えていたが、
有限変換では正則変換
\begin{equation}
    w\rightarrow w(z),\quad \bar{z} \rightarrow \bar{w}(\bar{z})
\end{equation}
が許される。
実際、線素の変わり方を見ると
\begin{equation}
    ds^2 = dz d\bar{z} \rightarrow dw d\bar{w} = \qty(\dv{w}{z})\qty(\dv{\bar{w}}{\bar{z}})dz d\bar{z}
\end{equation}
より計量が定数倍のみ変わっているのがわかる。
ここまでの結果を微分幾何っぽく表現すると、
2次元の無限小共形変換は正則なベクトル場であり、
有限の共形変換は正則な微分同相であるということができる。

局所的に成り立つコーシーリーマンの関係式から変換を見ていったが、
これが大域的、つまりリーマン面全体で使えるような変換はどのようなものかを見ていく。
このとき、変換の生成子はどのようなものかというのを考えるのが筋である。
座標の無限小変換のうち、
\begin{equation}
    z \rightarrow z' = z + \epsilon(z),\quad \bar{z} \rightarrow \bar{z}' = \bar{z} + \bar{\epsilon}(\bar{z})
\end{equation}
の形のものが共形変換を生成することが分かった。
ここで、無限小変換のパラメータを
\begin{align}
    \epsilon(z) &= \sum_{n=-\infty}^{\infty}c_n z^{n+1}\\
    \bar{\epsilon}(\bar{z}) &= \sum_{n=-\infty}^{\infty}\bar{c}_n \bar{z}^{n+1}
\end{align}
のように座標で展開してみる。
すると座標変換は
\begin{equation*}
    z \rightarrow z' = \qty(1 - \sum_{n=-\infty}^{\infty}c_n \underlineset{=:l_n}{(-z^{n+1}\partial)} )z
\end{equation*}
というようになる。
これより生成子を
\begin{equation}
    l_n =  -z^{n+1}\partial,\quad \bar{l}_n = -\bar{z}^{n+1}\bar{\partial}
\end{equation}
というように導入できる。この生成子はある意味複素ベクトル場とみなすことができる。
生成子を導入したので交換関係を書くと、
\begin{align}
    [l_m,\,l_n] &= (m-n)l_{m+n}\\
    [\bar{l}_m,\,\bar{l}_n] &= (m-n) \bar{l}_{m+n}\\
    [l_m,\,\bar{l}_n] &= 0
\end{align}
となる。
実際、
\begin{align*}
    [l_m,\,l_n] &= [z^{m+1}\partial,\,z^{n+1}\partial] = z^{m+1}[\partial,\,z^{n+1}]\partial + z^{n+1}[z^{m+1},\,\partial]\partial\\
    &=(n+1)z^{m+n+1}\partial -(m+1)z^{m+n+1}\partial = (m-n)l_{m+n}
\end{align*}
である。
リー代数を考えるときには交換関係のみが大事になる。
この変形を見てわかるように無限小パラメータの\(\epsilon=\sum c_n z^{n+1}\)の\(z\)の次数は適当に決めてもよさそうである。
なので\(\epsilon= \sum c_n z^{n+10}\)とかにしても問題ないように見える。

3次元以上では並進・回転・拡大・反並反操作の4種類であった生成子が
2次元では加算無限個あることがわかる。つまり対称性がとんでもなく広がっていると言える。
生成子が無限個あることから無限次元のリー代数を作ることになる。
そういった点が数学的にも注目される点である。
また、正則関数\(z\)と反正則関数\(\bar{z}\)が独立になっているのも重要な点になっている。

では無限小変換から大域的な変換に直していこう。
生成子が\(z=0\)近傍で正則であるためには\(n\geq -1\)である必要があるのがわかる。
次に無限遠点\(z=\epsilon\)近傍で正則であるのを確かめる必要がある。
その条件を調べるには\(w=1/z\)というように座標変換をして\(w=0\)周りのふるまいを見ればよい。
\begin{equation}
    l_n = -z^{n+1}\partial_z = -\qty(\frac{1}{w})^{n+1}\qty(\dv{w}{t})\partial_w = \qty(\frac{1}{w})^{n-1}\partial_w
\end{equation}
これより\(n\leq 1\)であればよいことがわかる。
リーマン球でこれ以外に特異点はないと考えれる。
なので大域的に定義できるのは\(n=-1,\,0,\,1\)の場合である。
しかもこれらは部分代数となっている。この部分代数を大域的共形代数という。
この部分代数の生成子の具体的な表式を見てみよう。
まずは\(n=-1\)のとき、
\begin{equation*}
    l_{-1} = -\partial,\quad \bar{l}_{-1} = -\bar{\partial}.
\end{equation*}
これはつまり、複素平面でいうところの\(\theta=\pm \pi/4\)度方向への並進操作に対応している
\footnote{光円錐座標だと光の世界線上の移動?}。
座標変換の式で表すと
\begin{equation*}
    z\rightarrow z' = z + b.
\end{equation*}

次に\(n=0\)のとき、
\begin{align*}
    &l_0 = -z\partial,\quad \bar{l}_0 = -\bar{z}\bar{\partial}\\
    &\quad \rightarrow l_0 + \bar{l}_0 = x^1\partial_1 + x^2\partial_2\\
    &\quad \rightarrow i(l_0-\bar{l}_0) = x^1\partial_2 -x^2\partial_1
\end{align*}
となる。これはつまり拡大と回転を表している。
座標変換の式で表すと
\begin{equation*}
    z\rightarrow z'= az.
\end{equation*}

では\(l=2\)のとき、
\begin{align*}
    &l_{1} = -z^2\partial,\quad \bar{l}_{1} = -z^2\bar{\partial}\\
    &\quad \rightarrow l_1 + \bar{l}_1 = (x^{(1)2}-x^{(2)2})\partial_1 - 2x^1x^2\partial_2\\
    &\quad \rightarrow i(l_1-\bar{l}_1) = -(x^{(1)2}-x^{(2)2})\partial_2 + 2x^1x^2\partial_1
\end{align*}
となる。これは特殊共形変換(2.25)の生成子と同じになっている。%: TODO どう対応しているか示す。
これは座標変換の式で表すと
\begin{equation*}
    \frac{1}{z'} = \frac{1}{z} + c \rightarrow z' = \frac{z}{cz+1}
\end{equation*}
と書くことができる。
これらの生成子から生成される無限小変換によって座標は
\begin{equation}
    z \rightarrow z' = \frac{az+b}{cz +d}\quad (ad-bc=1)
\end{equation}
と変換されるのがわかる。
また、\(z=z_1/z_2\)というように\(z\)が射影空間の元だということを明記するとこの変換は
\begin{equation}
    \begin{pmatrix}
        z_1\\ z_2
    \end{pmatrix}
    \rightarrow
    \begin{pmatrix}
        z'_1 \\z'_2
    \end{pmatrix}
    = \begin{pmatrix}
        a & b\\
        c & d
    \end{pmatrix}
    \begin{pmatrix}
        z_1 \\ z_2
    \end{pmatrix}
    \quad (ad-bc=1)
\end{equation}
というように書くことができる。
つまり射影空間\(CP\)に\(SL(2,\,\mathbb{C})/\mathbb{Z}_2\)が作用していると言える。
ただしこの\(\mathbb{Z}_2\)は反転操作\((a,\,b,\,c,\,d)\rightarrow(-a,\,-b,\,-c,\,-d)\)をしても作用が変わらないことを表している。

\section{プライマリ場と相関関数}
2次元の共形変換に関して、場がどう変わるかを調べる。
座標変換\(z\rightarrow w(z)\)のもとで場が
\begin{equation}
    \mathcal{O}'(w,\,\bar{w}) = \qty(\dv{z}{w})^{h}\qty(\dv{\bar{z}}{\bar{w}})^{\bar{h}}\mathcal{O}(z,\,\bar{z})
\end{equation}
というように変換するものに注目する。
これは
\begin{equation*}
    \mathcal{O}_{z_1\cdots z_h \bar{z}_1\cdots \bar{z}_{\bar{h}}}dz^1\cdots dz^h d\bar{z}^1\cdots d\bar{z}^{\bar{h}}
\end{equation*}
というテンソル場と同じ変換をしている。
このパラメータ\((h,\,\bar{h})\)は共形ウェイトと呼ばれ、
共形次元を\(\Delta\), 2次元回転のスピンを\(s\)としたとき、
\begin{equation}
    \Delta =h + \bar{h},\quad s = h - \bar{h}
\end{equation}
で関係づけられる。
このことはスケール変換\(w=\lambda z\)と回転変換\(w=e^{i\theta}z\)に対して
\begin{align}
    \mathcal{O}'(\lambda z,\,\lambda\bar{z}) &= \lambda^{-(h + \bar{h})}\mathcal{O}(z,\,\bar{z})\\
    \mathcal{O}'(e^{i\theta} z,\,e^{-i\theta}\bar{z}) &= e^{-i(h-\bar{h})\theta}\mathcal{O}(z,\,\bar{z})
\end{align}
というように変換することからわかる。
(3.21)式の分数変換で与えられる大域的共形変換において、(3.23)式を満たす場を準プライマリ場という。
一方、任意の正則関数\(w(z)\)に関して成り立つ場合にはプライマリ場と呼ぶ。
この定義より、プライマリ場ならば準プライマリ場であるが、準プライマリ場ならばプライマリ場とは限らない。

プライマリ場は局所共形変換に対する変換で特徴づけることができる。
無限小変換\(z\rightarrow z' = z - \epsilon(z)\)を考える。
空間そのものが共形変換して、座標と場両方とも変化して生じる差分
\(\delta\mathcal{O}=\mathcal{O}'(w,\,\bar{w})-\mathcal{O}(z,\,\bar{z})\)は
\begin{align*}
    \mathcal{O}'(w,\,\bar{w})
    &= (1-\partial\epsilon)^h(1-\bar{\partial}\bar{\epsilon})^{\bar{h}}\mathcal{O}(z,\,\bar{z})
    = \mathcal{O}(z,\,\bar{z}) +(\partial\epsilon+\bar{\partial}\bar{\epsilon})\mathcal{O}(z,\,\bar{z})\\
    &\rightarrow \delta \mathcal{O} = (\partial\epsilon+\bar{\partial}\bar{\epsilon})\mathcal{O}(z,\,\bar{z})
\end{align*}
である。
一方共形変換したとき、同じ位置における場の変化の差分
\(\bar{\delta}\mathcal{O}=\mathcal{O}'(z,\,\bar{z})-\mathcal{O}(z,\,\bar{z})\)は
\begin{align}
    \bar{\delta}\mathcal{O}
    &= \mathcal{O}'(z,\,\bar{z})-\mathcal{O}'(w,\,\bar{w}) + \mathcal{O}'(w,\,\bar{w}) -\mathcal{O}(z,\,\bar{z})\notag\\
    &= \mathcal{O}'(z,\,\bar{z}) -\qty(
        \mathcal{O}'(z,\,\bar{z}) - \epsilon\partial\mathcal{O}'(z,\,\bar{z})-\bar{\epsilon}\bar{\partial}\mathcal{O}'(z,\,\bar{z}))
        +\delta\mathcal{O}\notag\\
    &= \epsilon\partial\mathcal{O}-\bar{\epsilon}\bar{\partial}\mathcal{O}
    + (\partial\epsilon+\bar{\partial}\bar{\epsilon})\mathcal{O}(z,\,\bar{z})\notag\\
    &= (h\partial+\epsilon \partial)\mathcal{O}(z,\,\bar{z}) +(\bar{h}\bar{\partial}+\bar{\epsilon} \bar{\partial})\mathcal{O}(z,\,\bar{z})
\end{align}
というようになる。同じ位置における場の変換則によって場を特徴づけできるのでこれをプライマリ場の定義式とみなすことができる
\footnote{江口読んでると何か勘違いしてるような気がして来る}。
またこの式を見るとプライマリ場は共形ウェイト\((h,\,0)\)を持つ正則部分と
\((0,\,\bar{h})\)を持つ反正則部分に因子化しているのがわかる。
つまり、
\begin{equation*}
    \mathcal{O}_{h,\,\bar{h}}(z,\,\bar{z}) = \mathcal{O}_{h}(z)\mathcal{O}_{\bar{h}}(\bar{z})
\end{equation*}
なのでこれを使うと大域的共形変換に対する相関関数の制約は2章でやった結果をそのまま流用できるのがわかる。
残りこの節は割愛。
\setcounter{equation}{40}

\section{共形 Ward-高橋恒等式}
Ward-高橋恒等式のデルタ関数を留数定理による表現に変えてみる。
目標は
\begin{equation}
    \delta(z,\,\bar{z}) = \frac{1}{\pi}\bar{\partial}\frac{1}{z},\quad
    \delta(z,\,\bar{z}) = \frac{1}{\pi}\partial\frac{1}{\bar{z}}
\end{equation}
である。ここで左辺のデルタ関数は\(z=0,\,\bar{z}=0\)のときの値を返すものである。
そのためにグリーンの定理
\begin{equation}
    \int_R d^2x \qty(-\pdv{v^1(x^1,\,x^2)}{x^2}+\pdv{v^2(x^1,\,x^2)}{x^1})
    = \int_{\partial R} v^1(x^1,\,x^2)dx^1+v^2(x^1,\,x^2)dx^2
\end{equation}
を複素関数に利用する。
1次元の複素数の 1-form 等は
\begin{equation}
    dz = dx^1+idx^2,\quad d\bar{z} = dx^1-idx^2
\end{equation}
\begin{align*}
    &\partial_1 = \partial +\bar{\partial},& &\partial_2 = i(\partial-\bar{\partial})\\
    &v^1 = \frac{v+\bar{v}}{2}, & & v^2 = \frac{v-\bar{v}}{2i}
\end{align*}
であるので、複素微分形式に対するグリーンの定理は
\begin{equation}
    \int_R dz d\bar{z} (\partial v - \bar{\partial}\bar{z}) = -\frac{1}{2i}\int_{\partial R}(vd\bar{z}+\bar{v}dz)
\end{equation}
というようにできる。

ある正則な関数\(f(z)\)を用意して、領域\(R\)を\(z=0\)を含むようにとると
\setcounter{equation}{45}
\begin{equation}
    \frac{1}{\pi}\int_R dz d\bar{z} f(z)\bar{\partial}\frac{1}{z}
    = \frac{1}{\pi}\int_R dz d\bar{z} \bar{\partial}\qty(\frac{f(z)}{z} )
    = \frac{1}{2\pi i}\int_{\partial R} dz\frac{f(z)}{z} = f(0)
\end{equation}
というようにしてデルタ関数と同じ役割をすることを示せた。

ではこれを Ward-高橋恒等式に入れてみる。
\(T^{12}\)から\(T^{z\bar{z}}\)の変換も忘れずに整理していく。
\begin{align}
    &\partial_\mu\ev{T^{\mu\nu}\mathcal{O}_1\cdots\mathcal{O}_n}
             = - \sum_{j=1}^n \delta(x-x_j)\ev{\mathcal{O}_1\cdots[\partial_j^\nu\mathcal{O}_j]\cdots\mathcal{O}_n}\notag\\
    &\partial\ev{T^{z\nu}\mathcal{O}_1\cdots\mathcal{O}_n}
        +\bar{\partial}\ev{T^{\bar{z}\nu}\mathcal{O}_1\cdots\mathcal{O}_n}
             = - \sum_{j=1}^n \delta(x-x_j)\partial_j^\nu\ev{\mathcal{O}_1\cdots\mathcal{O}_n}\notag
\end{align}
この式を\((\nu=1\text{のとき})-i(\nu=2\text{のとき})\)というようにすると
\begin{align}
    &\partial\ev{T^{z\bar{z}}\mathcal{O}_1\cdots\mathcal{O}_n}
        +\bar{\partial}\ev{T^{\bar{z}\bar{z}}\mathcal{O}_1\cdots\mathcal{O}_n}
             = - \sum_{j=1}^n \delta(z-z_j,\,\bar{z}-\bar{z}_j)2\partial_j\ev{\mathcal{O}_1\cdots\mathcal{O}_n}\notag\\
    &2\pi\partial\ev{T_{\bar{z}z}\mathcal{O}_1\cdots\mathcal{O}_n}
        +2\pi\bar{\partial}\ev{T_{zz}\mathcal{O}_1\cdots\mathcal{O}_n}
            = - \sum_{j=1}^n \bar{\partial}\qty(\frac{1}{z-z_j})\partial_j\ev{\mathcal{O}_1\cdots\mathcal{O}_n}
\end{align}
また\((\nu=1\text{のとき})+i(\nu=2\text{のとき})\)というようにすると
\begin{align}
    &\partial\ev{T^{zz}\mathcal{O}_1\cdots\mathcal{O}_n}
        +\bar{\partial}\ev{T^{\bar{z}z}\mathcal{O}_1\cdots\mathcal{O}_n}
             = - \sum_{j=1}^n \delta(z-z_j,\,\bar{z}-\bar{z}_j)2\bar{\partial}_j\ev{\mathcal{O}_1\cdots\mathcal{O}_n}\notag\\
    &2\pi\partial\ev{T_{\bar{z}\bar{z}}\mathcal{O}_1\cdots\mathcal{O}_n}
        +2\pi\bar{\partial}\ev{T_{z\bar{z}}\mathcal{O}_1\cdots\mathcal{O}_n}
            = - \sum_{j=1}^n \partial\qty(\frac{1}{\bar{z}-\bar{z}_j})\bar{\partial}_j\ev{\mathcal{O}_1\cdots\mathcal{O}_n}
\end{align}
というようにできる。

回転変換に関しては
\begin{align}
    &2\ev{T_{\bar{z}z}\mathcal{O}_1\cdots\mathcal{O}_n}-\ev{T^{z\bar{z}}\mathcal{O}_1\cdots\mathcal{O}_n}
        = - \sum_{j=1}^n \delta(z-z_j,\,\bar{z}-\bar{z}_j)s_j\ev{\mathcal{O}_1\cdots\mathcal{O}_n}\\
    &\Sigma_{z\bar{z}}\mathcal{O}_{z_1\dots z_h\bar{z}_1\dots \bar{z}_{\bar{h}}}
    = -\frac{h-\bar{h}}{2}\mathcal{O}_{z_1\dots z_h\bar{z}_1\dots \bar{z}_{\bar{h}}}
    = -\frac{s}{2}\mathcal{O}_{z_1\dots z_h\bar{z}_1\dots \bar{z}_{\bar{h}}}
\end{align}
というようになる。 %: TODO 2次元回転変換とプライマリ場の導出

また、スケール変換に関しては
\begin{align}
    \ev{T\indices{^\mu_\mu}\mathcal{O}_1\cdots\mathcal{O}_n}
    &= - \sum_{j=1}^n \delta(z-z_j,\,\bar{z}-\bar{z}_j)\Delta_j\ev{\mathcal{O}_1\cdots\mathcal{O}_n}\notag\\
    2\ev{T_{\bar{z}z}\mathcal{O}_1\cdots\mathcal{O}_n} + 2\ev{T_{z\bar{z}}\mathcal{O}_1\cdots\mathcal{O}_n}
    &= - \sum_{j=1}^n \delta(z-z_j,\,\bar{z}-\bar{z}_j)\Delta_j\ev{\mathcal{O}_1\cdots\mathcal{O}_n}
\end{align}
となる。
回転変換とスケール変換の式(3.49)と(3.51)を組み合わせてデルタ関数を書き直すと
\begin{align}
    2\pi\ev{T_{\bar{z}z}\mathcal{O}_1\cdots\mathcal{O}_n} = -\sum_{j=1}^{n}\bar{\partial}\qty(\frac{1}{z-z_j})h_j\ev{\mathcal{O}_1\cdots\mathcal{O}_n}\\
    2\pi\ev{T_{z\bar{z}}\mathcal{O}_1\cdots\mathcal{O}_n} = -\sum_{j=1}^{n}\partial\qty(\frac{1}{\bar{z}-\bar{z}_j})\bar{h}_j\ev{\mathcal{O}_1\cdots\mathcal{O}_n}
\end{align}
となる。これを使うと並進変換の式(3.47),(3.48)から\(T_{\bar{z}z},\,T_{z\bar{z}}\)の項を消すことができる。
\begin{align}
    &\bar{\partial}\qty[
        \ev{T(z,\,\bar{z})\mathcal{O}_1\cdots\mathcal{O}_n}-\sum_{j=1}^{n}\qty(
            \frac{1}{z-z_j}\pdv{z_j} + \frac{h_j}{(z-z_j)^2})\ev{\mathcal{O}_1\cdots\mathcal{O}_n}]=0\\
    &\partial\qty[
        \ev{\bar{T}(z,\,\bar{z})\mathcal{O}_1\cdots\mathcal{O}_n}-\sum_{j=1}^{n}\qty(
            \frac{1}{\bar{z}-\bar{z}_j}\pdv{z_j} + \frac{\bar{h}_j}{(\bar{z}-\bar{z}_j)^2})\ev{\mathcal{O}_1\cdots\mathcal{O}_n}]=0
\end{align}
と書ける。
ここで
\begin{equation}
    T = -2\pi T_{zz},\quad \bar{T}=-2\pi T_{\bar{z}\bar{z}}
\end{equation}
とした。
また、この式を次のようにも書き換えておく。
\begin{align}
    &\ev{T(z,\,\bar{z})\mathcal{O}_1\cdots\mathcal{O}_n}=\sum_{j=1}^{n}\qty(
            \frac{1}{z-z_j}\pdv{z_j} + \frac{h_j}{(z-z_j)^2})\ev{\mathcal{O}_1\cdots\mathcal{O}_n}+\cdots\\
    &\ev{\bar{T}(z,\,\bar{z})\mathcal{O}_1\cdots\mathcal{O}_n}=\sum_{j=1}^{n}\qty(
            \frac{1}{\bar{z}-\bar{z}_j}\pdv{z_j} + \frac{\bar{h}_j}{(\bar{z}-\bar{z}_j)^2})\ev{\mathcal{O}_1\cdots\mathcal{O}_n} + \cdots
\end{align}
うしろの\(\cdots\)は\(z\rightarrow z_j\)で発散しない微分で消える\(z\)か\(\bar{z}\)の正則関数を表す。

\end{document}