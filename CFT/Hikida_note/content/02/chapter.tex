\documentclass[../../master.tex]{subfiles}

\graphicspath{{./image/}}

\begin{document}
\setcounter{chapter}{1}
\chapter{一般次元の共形場理論}
\section{一般次元の共形対称性}
一章にて、線素が併進と回転変換に関して不変であることを見た。
この節ではユークリッド計量の場合を考えるが、
ミンコフスキー計量のときであっても同様の議論が成り立つ。

共形変換を考える際にはこれまでの議論を少し拡張する必要がある。
一般の計量テンソル\(g_{\mu\nu}(x)\)を用意して、
線素を
\begin{equation}
    ds^2=g_{\mu\nu}(x)dx^\mu dx^\nu
\end{equation}
と書くことにする。
この計量テンソルは座標変換\(x\rightarrow x'\)のもとで
\begin{equation}
    g_{\mu\nu} \rightarrow g'_{\mu\nu}=\pdv{x^\rho}{x'^\mu} \pdv{x^\sigma}{x'^\nu}g_{\rho\sigma}
\end{equation}
のように変換する。
以下ではこの計量テンソルの変換性を利用することで、対称性の解析を行いたい。

\subsection*{並進・回転変換}
まずは計量テンソルを\(g_{\mu\nu}(x)=\delta{\mu\nu}\)というように固定する。
このとき、計量テンソルを不変にする変換が並進と回転変換と与えられることを見よう。
座標の無限小変換
\begin{equation}
    x^\mu\rightarrow x'^\mu = x^\mu - \epsilon^\mu(x)
\end{equation}
を考える。
この変換でのヤコビアンは次のような式を考えるとわかる。
\begin{align*}
    \pdv{x'^\mu}{x^\nu} &=\delta_\nu^\mu - \partial_\nu\epsilon^\mu\\
    \pdv{x^\rho}{x'^\sigma} &=\delta_\sigma^\rho + \partial_\sigma \epsilon^\rho
\end{align*}
よって計量テンソルは
\begin{align}
    g_{\mu\nu}\rightarrow g'_{\mu\nu}
    &=(\delta_\mu^\rho+\partial_\mu\epsilon^\rho)(\delta_\nu^\sigma+\partial_\nu\epsilon^\sigma)g_{\rho\sigma}\notag\\
    &=g_{\mu\nu} + \partial_\mu\epsilon_\nu + \partial_\nu\epsilon_\mu
\end{align}
最後の等号は\(\epsilon^2\)のオーダーの項を無視して、計量テンソルがユークリッド計量ことを使った。

したがって計量テンソルを不変に保つという条件は
\begin{equation}
    \partial_\mu\epsilon_\nu + \partial_\nu\epsilon_\mu = 0
\end{equation}
というのがわかった。(Killing 方程式)

この式に\(\partial_\rho\)を作用させると
\begin{equation}
    \partial_\rho\partial_\mu\epsilon_\nu + \partial_\rho\partial_\nu\epsilon_\mu
    =\partial_\mu\partial_\rho\epsilon_\nu + \partial_\nu\partial_\rho\epsilon_\mu
    =\partial_\mu(-\partial_\nu \epsilon_\rho)+\partial_\nu(-\partial_\mu \epsilon_\rho)
    =-2\partial_\mu\partial_\nu\epsilon_\rho =0
\end{equation}
これより、\(\epsilon\)は\(x\)の一次式
\begin{equation}
    \epsilon_\mu = a_\mu + b_{\mu\nu}x^\nu
\end{equation}
で書けることがわかる。
これを(2.5)式に戻すと
\begin{align*}
    0=\partial_\mu(a_\nu+b_{\nu\rho}x^\rho) + \partial_\nu(a_\mu+b_{\mu\rho}\rho)=b_{\mu\nu}+b_{\nu\mu}
\end{align*}
となり係数\(b_{\mu\nu}\)は反対称だとわかる。
よって座標変換は
\begin{equation*}
    x^\mu\rightarrow x'^\mu = x^\mu -a^\mu -b\indices{^{\mu}_\nu}x^\nu
\end{equation*}
というように書ける。つまり\(a^\mu\)は並進、\(b_\nu^\mu x^\nu\)は回転を表している。

無限小変換をリー群のように書きなおすと
\begin{equation*}
    x^\mu\rightarrow x'^\mu = (1-i\epsilon^\nu G_\nu)x^\mu.
\end{equation*}
つまり、並進の生成子\(P_\mu\)は
\begin{align}
    (1-ia^\nu P_\nu)x^\mu&=x^\mu-a^\mu\notag\\
    \rightarrow \quad P_\mu &= -i\partial_\mu
\end{align}
とわかる。
また、回転変換の生成子\(L_{\mu\nu}\)は
\begin{align}
    \qty(1-i\frac{b^{\nu\rho}}{2}L_{\nu\rho})x^\mu &= x^\mu - b\indices{^{\mu}_\sigma} x^\sigma \notag\\
    b^{\nu\rho}L_{\nu\rho}x^\mu &= -2ib\indices{^{\mu}_\sigma} x^\sigma\notag\\
    L_{\nu\rho}x^\mu&=-ib_{\nu\rho}^{-1}b^{\mu\sigma} x_\sigma\notag\\
    &=i(\delta_\nu^\mu\delta_\rho^\sigma-\delta_\nu^\sigma\delta_\rho^\mu)x_\sigma\notag\\
    &=i(x_\nu\partial_\rho-x_\rho\partial_\nu)x^\mu\notag\\
    \rightarrow \quad L_{\mu\nu} &= i(x_\mu\partial_\nu-x_\nu\partial_\mu)\\
    &=-(x_\mu P_\nu-x_\nu P_\mu)\notag
\end{align}
となる。

これらの生成子の交換関係は
\begin{equation}
    \begin{split}
        &[P_\mu,\,P_\nu]=0,\quad [P_\mu,\,L_{\rho\sigma}] = i(\eta_{\mu\rho}P_\sigma-\eta_{\mu\sigma}P_\rho)\\
        &[L_{\mu\nu},\,L_{\rho\sigma}] = i(\eta_{\nu\rho}L_{\mu\sigma}-\eta_{\mu\rho}L_{\nu\sigma}-\eta_{\nu\sigma}L_{\mu\rho}+\eta_{\mu\sigma}L_{\nu\rho})
    \end{split}
\end{equation}
となる。
実際、
\begin{align*}
    [P_\mu,\,L_{\rho\sigma}]
    &= [P_\mu,\,-(x_\rho P_\sigma-x_\sigma P_\rho)]\\
    &= i(\eta_{\mu\rho}P_\sigma-\eta_{\mu\sigma}P_\rho)\\
    [L_{\mu\nu},\,L_{\rho\sigma}]
    &=[x_\mu P_\nu-x_\nu P_\mu,\,x_\rho P_\sigma-x_\sigma P_\rho]\\
    &=[x_\mu P_\nu,\,x_\rho P_\sigma]-[x_\mu P_\nu,\,x_\sigma P_\rho]
        -[x_\nu P_\mu,\,x_\rho P_\sigma]+[x_\nu P_\mu,\,x_\sigma P_\rho]\\
\end{align*}
そして
\begin{align*}
    [x_\mu P_\nu,\,x_\rho P_\sigma]
    &=[x_\mu,\,x_\rho P_\sigma]P_\nu + x_\mu [P_\nu,\,x_\rho P_\sigma]\\
    &=i(\eta_{\mu\sigma}x_\rho P_\nu - \eta_{\rho\nu} x_\mu P_\sigma)
\end{align*}
より
\begin{align*}
    [L_{\mu\nu},\,L_{\rho\sigma}]
    &= i(
        \eta_{\mu\sigma}x_\rho P_\nu - \eta_{\rho\nu} x_\mu P_\sigma
        -\eta_{\mu\rho}x_\sigma P_\nu + \eta_{\sigma\nu} x_\mu P_\rho
        -\eta_{\nu\sigma}x_\rho P_\mu + \eta_{\rho\mu} x_\nu P_\sigma
        +\eta_{\nu\rho}x_\sigma P_\mu - \eta_{\sigma\mu} x_\nu P_\rho
    )\\
    &= i(
        -\eta_{\nu\rho}(x_\mu P_\sigma-x_\sigma P_\mu) + \eta_{\mu\rho}(x_\nu P_\sigma-x_\sigma P_\nu)
        +\eta_{\nu\sigma}(x_\mu P_\rho-x_\rho P_\mu) - \eta_{\mu\sigma}(x_\rho P_\nu-x_\nu P_\rho)
    )\\
    &=i(\eta_{\nu\rho}L_{\mu\sigma}-\eta_{\mu\rho}L_{\nu\sigma}-\eta_{\nu\sigma}L_{\mu\rho}+\eta_{\mu\sigma}L_{\nu\rho})
\end{align*}
で確認できる。
回転の無限小変換を繰り返すことで特殊直交群\(SO(d)\)を作る。
また、生成子はリー代数\(so(d)\)を成す。

\subsection*{共形変換}
並進・回転変換で見たように、
計量テンソルの変化を見ることでどのような変換があるのか決まる。
計量テンソルが座標変換\(x\rightarrow x'\)のもとで、
\begin{equation}
    g_{\mu\nu}\rightarrow g'_{\mu\nu}(x') = \Lambda(x)g_{\mu\nu}(x)
\end{equation}
の形まで変化してよいとする。
このような変化となる変換を共形変換という。

これはどのようなものかというと角度を保つ変換である。
内積により角度を決めると
\begin{equation*}
    \cos\theta = \frac{g_{\mu\nu}u^\mu v^\nu}{\sqrt{(g_{\mu\nu}u^\mu u^\nu)(g_{\mu\nu}v^\mu v^\nu)}}
\end{equation*}
これに共形変換を加えても値は変わらないのはわかる。

この変換を Killing 方程式を通して具体定期にどのような式となるものか見てみる。
始めの計量テンソルを平坦な時空\(g_{\mu\nu}=\eta_{\mu\nu}=\delta_{\mu\nu}\)であるものとする。
このとき、座標変換\(x^\mu\rightarrow x'^\mu = x^\mu - \epsilon^\mu(x)\)をすると計量テンソルは
\begin{align*}
    g_{\mu\nu}\rightarrow g'_{\mu\nu}
    &=(\delta_\mu^\rho+\partial_\mu\epsilon^\rho)(\delta_\nu^\sigma+\partial_\nu\epsilon^\sigma)g_{\rho\sigma}\notag\\
    &=g_{\mu\nu} + \partial_\mu\epsilon_\nu + \partial_\nu\epsilon_\mu
\end{align*}
となる。
よって
\begin{equation}
    \partial_\mu\epsilon_\nu + \partial_\nu\epsilon_\mu = f(x)g_{\mu\nu}
\end{equation}
としてやれば、
\begin{equation*}
    g'_{\mu\nu} = (1+f(x))g_{\mu\nu}
\end{equation*}
となり、共形変換となる。
ではこの\(f(x)\)とはどのような形をしているのだろうか?
(2.12)のトレースをとるつまり、両辺に\(\eta^{\nu\mu}\)を掛けて添え字について足し合わせると、
\(\eta^{\nu\mu}\eta_{\mu\nu}=d\)となるので、
\begin{equation}
    f(x) = \frac{2}{d}\partial_\mu\epsilon^\mu
\end{equation}
また、(2.12)の両辺を\(\partial_\rho\)を作用させて
\begin{equation}
    \partial_\rho\partial_\mu\epsilon_\nu + \partial_\rho\partial_\nu\epsilon_\mu = \partial_\rho f(x)g_{\mu\nu}
\end{equation}
そして左辺の\(\partial_\rho\epsilon_\mu,\,\partial_\rho\epsilon_\nu\)を(2.12)式を使って、
\(\epsilon\)の添え字を\(\rho\)に変えるように変形すると
\begin{equation}
    -2\partial_\mu\partial_\nu\epsilon_\rho = \partial_\rho f(x)g_{\mu\nu} - \partial_\mu f(x)g_{\nu\rho} -\partial_\nu f(x)g_{\rho\mu}.
\end{equation}
さらに両辺のトレースをとって
\begin{equation}
    -\partial^\mu\partial_\mu \epsilon_\rho = (d-2)\partial_\rho f(x)
\end{equation}
これより、\(\epsilon\)がある式には両辺にダランベルシアンを作用させることで\(f(x)\)に変えることができるのがわかる。
(2.12)の両辺に\(\partial^\rho\partial_\rho\)を作用させて
\begin{equation}
    (2-d)\partial_\mu\partial_\nu f(x) = \partial^\rho\partial_\rho f(x) g_{\mu\nu}
\end{equation}
これが\(f(x)\)の取る条件である。
ただ、このままだとわかりにくいので両辺のトレースをとって
\begin{equation}
    (d-1)\partial^\rho\partial_\rho f(x) =0
\end{equation}
となる。
これは\(d=1\)のときには\(f(x)\)には何も制約を与えないことを表している。
このことは 1 次元では角度が定義できないというのと関連しているそうである。
また、\(d=2\)ではそもそもトレースをとる前の(2.17)式の時点で左辺\(=0\)となり、
条件が緩くなっている。

\(d=2\)のときは次の章で調べることとして、\(d>2\)のときを調べてみる。
二階微分が\(0\)であるため
\begin{equation*}
    f(x) = \alpha + \beta_\mu x^\mu
\end{equation*}
という形になる。これを(2.15)に入れてみると
\begin{equation*}
    -2\partial_\mu\partial_\nu\epsilon_\rho = \beta_\rho g_{\mu\nu} - \beta_\mu g_{\nu\rho} -\beta_\nu g_{\rho\mu}.
\end{equation*}
両辺のトレースをとってもいいが、すでに右辺が位置によらない量になっていることに注目すると
\begin{equation}
    \epsilon_\mu = a_\mu + b_{\mu\nu}x^\nu + c_{\mu\nu\rho}x^\nu x^\rho
\end{equation}
というような2次の式で書けるといえる。
\(a_\mu\)は並進変換を表しているのがわかる。
また、2次の項では\(x^\nu\)と\(x^\rho\)は交換するので、\(c\)の後ろ2つの添え字は対称である。

(2.13)式にこれを入れると
\begin{equation*}
    f(x) = \frac{2}{d}(b_\mu^\mu+c\indices{^{\mu}_{\mu\nu}} x^\nu+c\indices{^{\mu}_{\nu\mu}} x^\nu)
\end{equation*}
よって(2.12)式は
\begin{equation*}
    b_{\mu\nu}+b_{\nu\mu}+ 2(c_{\mu\nu\sigma}+c_{\nu\mu\sigma})x^\sigma
    =\frac{2}{d}b_\rho^\rho g_{\mu\nu}+\frac{4}{d}g_{\mu\nu}c\indices{^{\rho}_{\rho\sigma}} x^\sigma
\end{equation*}
0次の部分を比較して
\begin{equation}
    b_{\mu\nu}+b_{\nu\mu} = \frac{2}{d}b_\rho^\rho g_{\mu\nu}
\end{equation}
という条件式が得られる。
ここで
\[
    b_{\mu\nu}=b_{\mu\nu}^A+b_{\mu\nu}^S=b_{\mu\nu}^A+\frac{1}{d}b_{\rho}^{\rho} g_{\mu\nu}
\]
のように対称な部分と反対称な部分に分けることができる。
反対称部分は回転変換に対応している。

対称部分について、この部分だけをもとの座標変換の式に入れると
\begin{align}
    x^\mu\rightarrow x'^\mu
    &= x^\mu - \frac{1}{d}b_{\rho}^{\rho} x^\mu\notag\\
    &=: x^\mu - b x^\mu\\
    &= (1 -ib\underset{D}{\underline{(-ix^\nu\partial_\nu)}})x^\mu \notag
\end{align}
となる。
この微分方程式を解くと変換パラメータを\(s\)として
\begin{equation*}
    x_s^\mu = x_0^\mu \exp(-bs)
\end{equation*}
となる。
つまりこれはスケール変換に対応している。

そして\(c_{\mu\nu\sigma}\)の項については\(b_{\mu\nu}\)を導出するのに使った(2.12)式ではなく、
(2.15)式について考えるとよくて、これに\(\epsilon\) と \(f(x)\) の表式を入れると
\begin{align}
    -4c_{\rho\mu\nu}
    &= \frac{4}{d}(c_{\sigma\rho}^\sigma g_{\mu\nu}-c_{\sigma\mu}^\sigma g_{\nu\rho}-c_{\sigma\nu}^\sigma g_{\rho\mu}) \notag\\
    c_{\rho\mu\nu} &= B_\mu g_{\nu\rho} + B_\nu g_{\rho\mu} - B_\rho g_{\mu\nu},\quad B_\rho = \frac{1}{d}c\indices{^\sigma_{\sigma\rho}}\\
    &= 2 B_\mu g_{\nu\rho} - B_\rho g_{\mu\nu}\notag
\end{align}
これより、2次の項による変換は
\begin{align}
    x^\mu\rightarrow x'^\mu
    &= x^\mu - 2(x^\nu B_\nu)x^\mu + B^\mu \abs{x}^2\\
    &= \Bigl(1 -iB^\rho\underlineset{K_\rho}{\Bigl\{-i(2x_\rho x^\nu\partial_\nu - \abs{x}^2\partial_\rho)\Bigr\}}\Bigr)x^\mu \notag
\end{align}
となる。これを特殊共形変換という。

共形変換の中にある無限小変換の生成子が出そろった。
\begin{alignat}{4}
    P_\mu &= -i\partial_\mu& &\notag\\
    D &= -ix^\nu\partial_\nu &&= x^\nu P_\nu\\
    L_{\mu\nu} &= i(x_\mu\partial_\nu-x_\nu\partial_\mu)&&=-(x_\mu P_\nu-x_\nu P_\mu)\notag\\
    K_\rho &= -i(2x_\rho x^\nu\partial_\nu - \abs{x}^2\partial_\rho)
    &&=2x_\rho D - \abs{x}^2 P_\rho
\end{alignat}
そしてこれらの生成子は次の交換関係を持つ
\begin{equation}
    \begin{split}
        [D,\,K_\mu] &= -iK_\mu, \quad
        [D,\,P_\mu] = iP_\mu, \\
        [K_\mu,\,P_\nu] &= 2i(\eta_{\mu\nu}D-L_{\mu\nu}), \quad
        [K_\rho,\,L_{\mu\nu}] = i(\eta_{\rho\mu} K_\nu -\eta_{\rho\nu}K_{\mu}),\\
        [P_\rho,\,L_{\mu\nu}] &= i(\eta_{\rho\mu}P_\nu-\eta_{\rho\nu}P_\mu),\\
        [L_{\mu\nu},\,L_{\rho\sigma}] &= i(\eta_{\nu\rho}L_{\mu\sigma}-\eta_{\mu\rho}L_{\nu\sigma}-\eta_{\nu\sigma}L_{\mu\rho}+\eta_{\mu\sigma}L_{\nu\rho}).
    \end{split}
\end{equation}
ここで触れていないものはすべて可換。
そしてこのような交換関係を満たす生成子によって構成される代数を共形代数という\footnote{なので本文はちょっと不正確}。

実際にこれらの生成子の組み合わせ10通りとに加え、\(x\)との交換関係を調べておこう。
交換関係のライプニッツ則
\begin{equation*}
    [ABC,\,D] = [A,\,D]BC + A[B,\,D]C + AB[C,\,D]
\end{equation*}
も使って、
\begin{align*}
    [x_\mu,\,P_\nu] &= i\eta_{\mu\nu}\\
    [P_\mu,\,P_\nu] &= 0\\
    [P_\rho,\,L_{\mu\nu}] &=i(\eta_{\rho\mu}P_\nu-\eta_{\rho\nu}P_\mu) \quad\text{(並進・回転の節でやった。)}\\
    [K_\mu,\,P_\nu]
        &= [2x_\mu D - x_\rho x^\rho P_\mu,\,P_\nu]\\
        &= 2i\eta_{\mu\nu} D +2ix_\mu P_\nu -i\eta_{\rho\nu}x^\rho P_\mu - i\delta_\nu^\rho x_\rho P_\mu\\
        &= 2i\eta_{\mu\nu} D +2i(x_\mu P_\nu - x_\nu P_\mu)\\
        &= 2i(\eta_{\mu\nu} D - L_{\mu\nu} )\\ \\
    [D,\,x_\mu] &= [x^\nu P_\nu,\,x_\mu] = -ix_\mu\\
    [D,\,P_\mu] &= [x^\nu P_\nu,\,P_\mu] = iP_\mu\\
    [D,\,D] &= [D,\,x^\mu P_\mu]=-ix^\mu P_\mu + ix^\mu P_\mu =0\\
    [D,\,L_{\mu\nu}]
        &= [D,\,-(x_\mu P_\nu -x_\nu P_\mu)]\\
        &= ix_\mu P_\nu - ix_\mu P_\nu - (\mu\leftrightarrow\nu)=0\\
    [D,\,K_\mu]
        &= [D,\,2x_\mu D - \abs{x}^2 P_\mu]\\
        &= -2ix_\mu D -[D,\,x_\rho] x^\rho P_\mu- x_\rho[D,\,x^\rho] P_\mu - \abs{x^2}[D,\,P_\mu]\\
        &= -2ix_\mu D + 2i\abs{x}^2 P_\mu - i\abs{x}^2 P_\mu\\
        &= -i(2x_\mu D - \abs{x}^2 P_\mu) = -iK_\mu\\ \\
    [x_\mu,\,K_\nu] &=[x_\mu,\,2x_\nu D - \abs{x}^2 P_\nu] = 2ix_\mu x_\nu - i\abs{x}^2\eta_{\mu\nu}\\
    \rightarrow&[x^\mu,\,K_\nu]P_\mu = -iK_\nu\\
    [K_\mu,\,K_\nu]&= \text{(あとでやるかも)}\\ %TODO: 示す
    [K_\rho,\,L_{\mu\nu}]&=\text{(あとでやるかも)}\\ \\
    [x_\rho,\,L_{\mu\nu}]&=[x_\rho,\,-(x_\mu p_\nu-x_\nu p_\mu)] = i(\eta_{\rho\mu}x_\nu-\eta_{\nu\rho}x_[\mu])\\
    [L_{\mu\nu},\,L_{\rho\sigma}]
    &= i(\eta_{\nu\rho}L_{\mu\sigma}-\eta_{\mu\rho}L_{\nu\sigma}-\eta_{\nu\sigma}L_{\mu\rho}+\eta_{\mu\sigma}L_{\nu\rho})
    \quad\text{(並進・回転の節でやった。)}\\
\end{align*}

また、共形代数の同型らしきものが見える形を導入する。
反対称行列\(J_{ab}=-J_{ba}\quad(a,\,b = -1,\,0,\,1,\dots,\,d)\)を導入し、
\begin{equation}
    \begin{split}
        J_{(-1)0} = D, \quad J_{(-1)\mu} = \frac{1}{2}(P_\mu-K_\mu),\\
        J_{0\mu} = \frac{1}{2}(P_\mu + K_\mu),\quad J_{\mu\nu} = L{\mu\nu}
    \end{split}
\end{equation}
とすると、
\begin{equation}
    [J_{ab},\,J_{cd}] = i(\eta_{bc}J_{ad}-\eta_{ac}J_{bd}-\eta_{bd}J_{ac}+\eta_{ad}J_{bc})
\end{equation}
という関係が成り立つことがわかる。% TODO: 確かめるべき
これは\(SO(1,\,d+1)\)の生成子のなす交換関係と同じであるため、
共形代数は\(SO(1,\,d+1)\)と同一視できる。

これまでの解析を有限の座標変換に拡張すると
\begin{align}
    x'^\mu &= x^\mu +a^\mu& &\text{(並進)}\\
    x'^\mu &= \lambda x^\mu& &\text{(スケール変換)}\\
    x'^\mu &= \Lambda\indices{^\mu_\nu}x^\nu& &\text{(回転変換)}\\
    x'^\mu &= \frac{x^\mu  + B^\mu\abs{x}^2}{1+2B^\nu x_\nu +\abs{B}^2\abs{x}^2}& &\text{(特殊共形変換)}
\end{align}
のようになる。
特殊共形変換は無限小変換から出すことは自分にはできなかったが、逆は示せた。
この有限変換の標識を変換のパラメータ\(B^\nu\)が小さいとして虹の項を無視して分母を分子に持っていくと
\begin{align*}
    \frac{x^\mu  + B^\mu\abs{x}^2}{1+2B^\nu x_\nu +\abs{B}^2\abs{x}^2}
    &\simeq \qty(x^\mu  + B^\mu\abs{x}^2)\qty(1-2B^\nu x_\nu)\\
    &\simeq x^\mu  + B^\mu\abs{x}^2 - 2B^\nu x_\nu x^\mu\\
    &= (1-2 B^\nu x_\nu + B^\nu \abs{x}^2 \partial_\nu ) x^\mu\\
    &= \Bigl(1 -iB^\rho\underlineset{K_\rho}{\Bigl\{-i(2x_\rho x^\nu\partial_\nu - \abs{x}^2\partial_\rho)\Bigr\}}\Bigr)x^\mu \notag
\end{align*}
となる。

ポアンカレ群では出てこなかったスケール変換と特殊共形変換について、
どのようなものかを見ておこう。
スケール変換では計量テンソルは
\begin{equation}
    g_{\mu\nu} \rightarrow g'_{\mu\nu} = \frac{1}{\lambda^2} g_{\mu\nu}
\end{equation}
となるのはすぐわかるだろう。
この本の描像では座標系ではなく物理的実体を動かす方の座標変換になっているので、
つまりものが大きくなったり小さくなったりする変換になっている。

特殊共形変換について、
これは反転の演算子を\(I\)とすると\(K_\mu = I P_\mu I\)のように表すことができる。
実際、反転変換は式ににすると
\begin{equation}
    x^\mu \rightarrow x'^\mu = \frac{x^\mu}{\abs{x}^2}
\end{equation}
という代物である。
見てわかるように\(x=0\)では素直には定義できないので無限小変換というのを考えるのができない。
おそらく、無限遠点をからめたリー群みたいな話になるのかもしれない。
実際に、反転→並進→反転という操作をやると
\begin{align}
    x^\mu
    &\rightarrow \frac{x^\mu}{\abs{x}^2}\notag\\
    &\rightarrow \frac{x^\mu}{\abs{x}^2} + B^\mu\notag\\
    &\rightarrow \dfrac{\dfrac{x^\mu}{\abs{x}^2} + B^\mu}{\qty(\dfrac{x^\nu}{\abs{x}^2} + B^\nu)\qty(\dfrac{x_\nu}{\abs{x}^2} + B_\nu)} \notag\\
    &=\frac{x^\mu  + B^\mu\abs{x}^2}{1+2B^\nu x_\nu +\abs{B}^2\abs{x}^2}
\end{align}
というように表すことができる。
このように特殊共形変換が\(K_\mu = I P_\mu I\)という操作であるというのがわかると、
変換群の加法性に関しては明らかであるのがわかる。
群論における共役作用を取り入れた共役類の計算に相当する気がする。

また、計量テンソルに関しては
\begin{equation}
    g_{\mu\nu} \rightarrow g'_{\mu\nu} = (1+2B^\rho x_\rho +\abs{B}^2\abs{x}^2)^2 g_{\mu\nu}
\end{equation}
となる。\footnote{誤植だった。}
実際ヤコビアンを計算すると\(B^\mu\)の一次の項まで残して計算するとわかる。
\begin{align*}
    \pdv{x'^\mu}{x^\nu}
    &=\pdv{x^\nu}\qty{\frac{x^\mu  + B^\mu\abs{x}^2}{1+2B^\rho x_\rho +\abs{B}^2\abs{x}^2}}\\
    &=\frac{1}{1+2B^\rho x_\rho +\abs{B}^2\abs{x}^2}\qty{
        \tensor{\delta}{^\mu_\nu}+2B^\mu x_\nu -2(x^\mu+2B^\mu\abs{x}^2)(B_\nu + \abs{B}^2x_\nu)
    }
    =\frac{\tensor{\delta}{^\mu_\nu}}{1+2B^\rho x_\rho +\abs{B}^2\abs{x}^2}
\end{align*}

\section{準プライマリ場とその変換性}
\subsection*{変換性}
共形変換に対する変換性によって場\(\mathcal{O}(x)\)の分類を行う。

まず回転変換でスカラーとして変換する場\(\mathcal{O}(x)\)を考える。%TODO: この要請がどのように効くのかしら。
ここで、共形変換のもとで
\begin{equation}
    \mathcal{O}(x)\rightarrow\mathcal{O}'(x') = \Lambda(x)^{\Delta/2}\mathcal{O}(x)
\end{equation}
のように変換するものを準プライマリ場(quasi primary field)と呼ぶ。
一般次元ではこれは準というのを付けずに単にプライマリ場と呼ぶそうだが、
2次元での用語にあわせるとこのようになるらしい。

スケール変換では\(\Lambda(x)=\lambda^{-2}\)であるので
\begin{equation}
    \mathcal{O}(x)\rightarrow\mathcal{O}'(\lambda x) = \lambda^{-\Delta}\mathcal{O}(x)
\end{equation}
というように変換することがわかる。
これより\(\Delta\)は質量次元に関係する物理量であることがわかるのでこれを共形次元と呼ぶ。

次に回転変換のもとでテンソルとして変換する場\(\mathcal{O}_{\mu_1\cdots\mu_l}\)を考える。
この場合、\(dx^\mu\)を使って
\begin{equation}
    \mathcal{O}_{\mu_1\cdots\mu_l}dx^{\mu_1}\cdots dx^{\mu_l}
\end{equation}
のように添え字をつぶすことで共形次元が\(\Delta -l\)のスカラー場として扱うことができる。
どういうことかというとこれをスケール変換するとよい。
スケール変換すると
\begin{align*}
    \mathcal{O}_{\mu_1\cdots\mu_l}(x)\,\,dx^{\mu_1}\cdots dx^{\mu_l}
    &\rightarrow \mathcal{O}'_{\mu_1\cdots\mu_l}(\lambda x)\,\, d(\lambda x)^{\mu_1}\cdots d(\lambda x)^{\mu_l}\\
    &= \lambda^{-\Delta} \mathcal{O}_{\mu_1\cdots\mu_l}(x)\,\,d(\lambda x)^{\mu_1}\cdots d(\lambda x)^{\mu_l}\\
    &= \lambda^{-(\Delta-l)} \mathcal{O}_{\mu_1\cdots\mu_l}(x)\,\, dx^{\mu_1}\cdots dx^{\mu_l}(x)
\end{align*}
となるからである。

したがって、スピン\(l\)を持つ準プライマリ場は
\begin{equation}
    \mathcal{O}_{\mu_1\cdots\mu_l}(x)
    \rightarrow \mathcal{O}'_{\mu_1\cdots\mu_l}(x')
    =\Lambda(x)^{(\Delta-l)/2}\pdv{x^{\nu_1}}{x'^{\mu_1}}\cdots\pdv{x^{\nu_l}}{x'^{\mu_l}} \mathcal{O}_{\nu_1\cdots\nu_l}(x)
\end{equation}
のように変換される。\footnote{スピン\(l\)とテンソルの足が\(l\)個あることの関連がわからない。}%TODO: 調べる、

準プライマリ場が無限小変換\(x'^\mu = x^\mu -\epsilon^\mu(x)\)のもとでどのように変換するか考える。
スカラー場の同じ座標点\(x\)での変化分は
\begin{align}
    \mathcal{O}(x)
    &\rightarrow \mathcal{O}'(x) = \mathcal{O}'(x'+\epsilon)\notag\\
    &=\Lambda(x+\epsilon)^{\Delta/2}\mathcal{O}(x+\epsilon)\notag\\
    &=(1+f(x))^{\Delta/2}(1 + \epsilon^\mu\partial_\mu)\mathcal{O}(x)\notag\\
    &=\qty(1+\frac{\Delta}{2}\frac{2}{d}\epsilon^\mu\partial_\mu)(1 + \epsilon^\mu\partial_\mu)\mathcal{O}(x)\notag\\
    &=\qty[1+\epsilon^\mu\partial_\mu +\frac{\Delta}{d}(\partial_\mu\epsilon^\mu)]\mathcal{O}(x)
\end{align}
となる。テンソル場に関しても同様に調べることができる。
\begin{align}
    \pdv{x^\nu}{x'^\mu}
    &= \tensor{\delta}{^\nu_\mu} + \partial^\nu \epsilon_\mu \notag\\
    &= \tensor{\delta}{^\nu_\mu} + \frac{1}{2}\partial^\nu \epsilon_\mu
        + \frac{1}{2} \qty(-\partial_\mu \epsilon^\nu +f(x)\tensor{\delta}{^\nu_\mu})\notag\\
    &= \qty{1+\frac{1}{d}(\partial^\rho \epsilon_\rho)}\tensor{\delta}{^\nu_\mu} + \frac{1}{2}\qty(\partial^\nu \epsilon_\mu
    -\partial_\mu \epsilon^\nu)
\end{align}
よってテンソル場の方の変換性は
\begin{align}
    \mathcal{O}_{\mu_1\cdots\mu_l}(x)
    &\rightarrow \mathcal{O}'_{\mu_1\cdots\mu_l}(x) \notag\\
    &= \mathcal{O}'_{\mu_1\cdots\mu_l}(x'+\epsilon) \notag\\
    &= \Lambda(x+\epsilon)^{(\Delta-l)/2}\pdv{x^{\nu_1}}{x'^{\mu_1}}\cdots\pdv{x^{\nu_l}}{x'^{\mu_l}} (1+\epsilon^\mu\partial_\mu)\mathcal{O}_{\nu_1\cdots\nu_l}(x) \notag\\
    &= \qty(1+\frac{\Delta-l}{d}\epsilon^\mu\partial_\mu)
        \qty[\qty{1+\frac{1}{d}(\partial^\rho \epsilon_\rho)}\tensor{\delta}{^{\nu_1}_{\mu_1}} + \frac{1}{2}\qty(\partial^{\nu_1} \epsilon_{\mu_1}
        -\partial_{\mu_1} \epsilon^{\nu_1})]\notag\\
        &\quad \cdots
        \qty[\qty{1+\frac{1}{d}(\partial^\rho \epsilon_\rho)}\tensor{\delta}{^{\nu_l}_{\mu_l}} + \frac{1}{2}\qty(\partial^{\nu_l} \epsilon_{\mu_l}
        -\partial_{\mu_l} \epsilon^{\nu_l})]
        (1+\epsilon^\mu\partial_\mu)\mathcal{O}_{\nu_1\cdots\nu_l}(x) \notag\\
    &= \mathcal{O}_{\mu_1\cdots\mu_l}(x) + \frac{l}{d}(\partial^\rho \epsilon_\rho)\mathcal{O}_{\mu_1\cdots\mu_l}(x)
        +\frac{1}{2}\sum_{j=1}^l \qty(\partial^{\nu_j} \epsilon_{\mu_j}-\partial_{\mu_j} \epsilon^{\nu_j})\mathcal{O}_{\mu_1\cdots\mu_{j-1}\nu_j\mu_{j+1}\cdots\mu_l}(x)
        + \epsilon^\mu\partial_\mu\mathcal{O}_{\mu_1\cdots\mu_l}(x) \notag\\
        &\quad + \frac{\Delta-l}{d} \epsilon^\mu \partial_\mu \mathcal{O}_{\mu_1\cdots\mu_l}(x)\notag\\
    &= \qty[1+\epsilon^\mu\partial_\mu +\frac{\Delta}{d}(\partial_\mu\epsilon^\mu)]\mathcal{O}_{\mu_1\cdots\mu_l}(x)
        -\frac{1}{2}\sum_{j=1}^l \qty(\partial_{\mu_j} \epsilon^{\nu_j}-\partial^{\nu_j} \epsilon_{\mu_j})\mathcal{O}_{\mu_1\cdots\mu_{j-1}\nu_j\mu_{j+1}\cdots\mu_l}(x)
\end{align}
を得ることができる。
\begin{equation*}
    \epsilon_\mu = a_\mu + b_{\mu\nu}x^\nu + c_{\mu\nu\rho}x^\nu x^\rho \tag{2.10}
\end{equation*}
を入れて考えてみる。
\(a_\mu\)のみのときそれは平行移動を表している。
\(\epsilon_\mu = a_\mu\)を(2.43)に入れて
\begin{align}
    \mathcal{O}'_{\mu_1\cdots\mu_l}(x)
    &= (1 + a^\mu \partial_\mu)\mathcal{O}_{\mu_1\cdots\mu_l}(x)\notag\\
    &= (1 +i a^\mu \underlineset{\mathscr{P}_\mu}{(-i\partial_\mu)})\mathcal{O}_{\mu_1\cdots\mu_l}(x) \notag\\
    \mathscr{P}_\mu &:= -i\partial_\mu
\end{align}
というように準プライマリ場対する並進変換の生成子\(\mathscr{P}_\mu\)が得られた。

次に\(b_{\mu\nu}\)の対称成分だけ取り出してスケール変換をしてみるとする。
すると\(\epsilon_\mu = b \eta_{\mu\nu}x^\nu\)を入れて
\begin{align}
    \mathcal{O}'_{\mu_1\cdots\mu_l}(x)
    &= (1 + bx^\mu \partial_\mu + b \Delta)\mathcal{O}_{\mu_1\cdots\mu_l}(x)
        -\frac{b}{2}\sum_{j=1}^l \underlineset{=0}{\qty(\partial_{\mu_j} \tensor{\delta}{^{\nu_j}_\rho}x^\rho-\partial^{\nu_j} \tensor{\eta}{_{\nu_j}_\rho}x^\rho)}\mathcal{O}_{\mu_1\cdots\mu_{j-1}\nu_j\mu_{j+1}\cdots\mu_l}(x)\notag\\
    &= \qty[1 +ib\underlineset{\mathscr{D}}{\Bigl\{-i(x^\mu\partial_\mu +\Delta)\Bigr\}}]\mathcal{O}_{\mu_1\cdots\mu_l}(x) \notag\\
    \mathscr{D} &:= -i(x^\mu\partial_\mu +\Delta).
\end{align}
というように準プライマリ場に対するスケール変換の生成子\(\mathscr{D}\)が得られた。

そして\(b_{\mu\nu}\)の反対称成分だけにして回転変換をしてみる。
すると\(\epsilon_\mu = b_{\mu\nu}x^\nu\)を入れて、
\begin{align}
    \mathcal{O}'_{\mu_1\cdots\mu_l}(x)
    &=\qty[1+b\indices{^\mu_\nu}x^\nu\partial_\mu +\frac{\Delta}{d}(\partial_\mu b\indices{^\mu_\nu}x^\nu)]\mathcal{O}_{\mu_1\cdots\mu_l}(x)
        -\frac{1}{2}\sum_{j=1}^l \qty(\partial_{\mu_j} \tensor{b}{^{\nu_j}_\nu}x^\nu-\partial^{\nu_j} b_{{\mu_j}\nu}x^\nu)\mathcal{O}_{\mu_1\cdots\mu_{j-1}\nu_j\mu_{j+1}\cdots\mu_l}(x)\notag\\
    &=\qty[1+b^{\mu\nu}x_\nu\partial_\mu]\mathcal{O}_{\mu_1\cdots\mu_l}(x)
        -\frac{1}{2}\sum_{j=1}^l \qty(\tensor{b}{^\mu^\nu}\tensor{\delta}{^{\nu_j}_\mu}\partial_{\mu_j} x_\nu-b\indices{^\mu^\nu}\eta_{\mu \mu_j}\partial^{\nu_j} x_\nu)\mathcal{O}_{\mu_1\cdots\mu_{j-1}\nu_j\mu_{j+1}\cdots\mu_l}(x)\notag\\
    &=\qty[1+\frac{b^{\mu\nu}}{2}(x_\nu\partial_\mu-x_\mu\partial_\nu)]\mathcal{O}_{\mu_1\cdots\mu_l}(x)
        -\frac{b^{\mu\nu}}{2}\underlineset{\Sigma_{\mu\nu}\mathcal{O}_{\mu_1\cdots\mu_l}(x)}{\sum_{j=1}^l \qty(\eta_{{\mu_j}\nu}\tensor{\delta}{^{\nu_j}_\mu}-\eta_{\mu \mu_j}\tensor{\delta}{^{\nu_j}_\nu})\mathcal{O}_{\mu_1\cdots\mu_{j-1}\nu_j\mu_{j+1}\cdots\mu_l}(x)}\notag\\
    &=\qty[1-\frac{b^{\mu\nu}}{2}\qty(x_\mu\partial_\nu -x_\nu\partial_\mu+\Sigma_{\mu\nu})]\mathcal{O}_{\mu_1\cdots\mu_l}(x)\notag\\
    &=\qty[1+i\frac{b^{\mu\nu}}{2}\underlineset{\mathscr{L}_{\mu\nu}}{\Bigl\{i(x_\mu\partial_\nu-x_\nu\partial_\mu + \Sigma_{\mu\nu})\Bigr\}}]\mathcal{O}_{\mu_1\cdots\mu_l}(x)\notag\\
    \mathscr{L}_{\mu\nu}&=i(x_\mu\partial_\nu-x_\nu\partial_\mu + \Sigma_{\mu\nu})
\end{align}
というようにして準プライマリ場に対する回転変換の生成子\(\mathscr{L}_{\mu\nu}\)が得られた。

最後に\(c_{\mu\nu\rho}\)の部分だけを取り出して特殊共形変換をするというのもよいが、
反転と並進を組み合わせて生成子を作ろう? %TODO:

これより
\begin{equation}
    \mathscr{K}_\mu = -i(2x_\mu x^\nu \partial_\nu -\abs{x}^2 \partial_\mu + 2\Delta x_\mu +2 x^\nu \Sigma_{\mu\nu})
\end{equation}
となる。

これらの変換の様子を調べるため、局所座標系の原点\(x=0\)周りでの場の変換を見ると
\setcounter{equation}{48}
\begin{align}
    \mathscr{P}_\mu\mathcal{O}(0) &= -i\partial_\mu\mathcal{O}(0)\\
    \mathscr{D}\mathcal{O}(0) &= -i\Delta\mathcal{O}(0)\\
    \mathscr{L}_{\mu\nu}\mathcal{O}(0) &= i\Sigma_{\mu\nu}\mathcal{O}(0)\\
    \mathscr{K}_\mu\mathcal{O}(0) &= 0
\end{align}
となる。した3つは右辺にかかっているのがある定数の固有値方程式の形になっている。

\subsection*{準プライマリ場の特徴づけ}
(2.52)式は準プライマリ場の定義とすることもできる。
つまり反転して平行移動してまた反転しても変わらないような場というのがプライマリ場であるといえる。
実際に構成していこう。
スケール変換と回転変換は交換するので同時固有状態をとってくることができる。
なので適宜規定をとってくれば(2.50)と(2.51)を満たすように\(\mathcal{O}(0)\)の基底をとることができる。
また場は演算子であるため、位置依存性は
\begin{equation}
    \mathcal{O}(x) = e^{ix^\rho \mathscr{P}_\rho} \mathcal{O}(0) e^{-ix^\rho \mathscr{P}_\rho}
\end{equation}
というようにして導入できる。

\subsection*{ディセンダント場\(^*\)}
無限小変換の生成子による定義
\setcounter{equation}{58}
\begin{equation}
    \mathscr{K}_\mu\mathcal{O}(0) = 0
\end{equation}
について考えてみよう。
準プライマリ場に\(\mathscr{P}_\mu\)を何回か作用していってできる場は
準プライマリ場ではない。
実際、
\begin{equation}
    \mathscr{K}_\mu \mathscr{P}_\nu \mathcal{O}(0) = [\mathscr{K}_\mu,\, \mathscr{P}_\nu] \mathcal{O}(0) = 2(\eta_{\mu\nu} \Delta + \Sigma_{\mu\nu})\mathcal{O}(0) \neq 0
\end{equation}
である。
また、\(x\rightarrow x' = \lambda x\)のスケール変換のものでは
\begin{equation}
    \pdv{x'^\mu} \mathcal{O}'(x') = \lambda^{-(\Delta+1)}\pdv{x^\mu} \mathcal{O}(x)
\end{equation}
と変化することより、場\(\mathscr{P}_\mu\mathcal{O}(0)\)の共形次元は\(\Delta+1\)となる。
同様に場\(\mathscr{P}_{\mu_1}\cdots\mathscr{P}_{\mu_n}\mathcal{O}(0)\)の共形次元は
\(\Delta+n\)である。
このことは\([D,\,\mathscr{P}_\mu]=i\mathscr{P}_\mu\)を用いても示すことができる。
つまり、\([D,\,\mathscr{K}_\mu]=-i\mathscr{K}_\mu\)より\(\mathscr{K}_\mu\)を場に作用させると
共形次元を1つ下げることができる。
このように\(\mathscr{P}_\mu\)や\(\mathscr{K}_\mu\)を\(\mathcal{O}(0)\)に作用させてできる場を
ディセンダント場という。

ここで理論が物理的に正常である、すなわちユニタリ性を保つように要請すると
共形次元\(\Delta\)に加減がついて
\begin{align}
    \Delta\geq d/2-1 \quad(l=0),\\
    \Delta\geq l+d-2 \quad(l=1,\,2,\dots)
\end{align}
となることが知られている。

ここまでの議論を表現論っぽく言うと、プライマリ場は最低ウェイト状態であり、
一般の場はプライマリ場に\(\mathscr{P}_\mu\)を作用させて構成できる。
プライマリ場\(\mathcal{O}(x)\)とその微分をまとめて共形族という。

ウェイトという言葉が出てきた。
これは\(d\)次元共形代数は\(SO(1,d+1)\)代数と同型であるため、
こちらの言葉を使ったものである。
\(SO(3)\)に対する全角運動量のように、(2.27)の各生成子と可換になる
Casimir 演算子というのも考えれる。
共形 Casimir 演算子は次で定められる。
\begin{equation*}
    J^2 = \frac{1}{2} J_{ab}J^{ab} = \frac{1}{2}M_{\mu\nu}M^{\mu\nu} - D^2 + \frac{1}{2}(P_\mu K^\mu + K^\mu P_\mu)
\end{equation*}
これは共形代数のすべての演算子と交換する。
この Casimir 演算子は
\begin{equation*}
    J^2 = \frac{1}{2}M_{\mu\nu}M^{\mu\nu} - D(D-d) + P_\mu K^\mu
\end{equation*}
と変形できる。
これをプライマリ場に作用させると
\begin{equation*}
    J^2 \mathcal{O}(0) = -\qty(\frac{1}{2}\Sigma_{\mu\nu}\Sigma^{\mu\nu}+\Delta(\Delta-d))\mathcal{O}(0)
\end{equation*}
というようになる。
\(\mathcal{O}(0)\)をスピン\(l\)のテンソル
\footnote{\(l\)階の対称トレースレステンソルのこと。}
とすれば
\begin{equation*}
    \frac{1}{2}\Sigma_{\mu\nu}\Sigma^{\mu\nu} \mathcal{O}(0) = -l(l+d-2)\mathcal{O}(0)
\end{equation*}
となるらしいので
\begin{equation*}
    J^2\mathcal{O}(0) = -\lambda_{\Delta,\,l}\mathcal{O}(0),\quad \lambda_{\Delta,\,l} = l(l+2-2)+\Delta(\Delta-d)
\end{equation*}
となる。

\section{相関関数への制約}
準プライマリ場の相関関数を調べる。
簡単のため、スカラー場についてのみ取り扱うことにする。
準プライマリ場は
\setcounter{equation}{63}
\begin{equation*}
    \mathcal{O}(x)\rightarrow\mathcal{O}'(x') = \mathcal{O}(x')=\Lambda(x)^{\Delta/2}\mathcal{O}(x)
\end{equation*}
というようにヤコビアンが付かずに変換する。
\(\Lambda(x)\)自体は場の関数ではないので期待値をすり抜けることに気付くと、
このときの準プライマリ場の相関関数は
\begin{align}
    \ev{\mathcal{O}'_1(x'_1)\cdots\mathcal{O}'_n(x'_n)}
    &= \ev{\Lambda(x_1)^{\Delta_1/2}\mathcal{O}_1(x_1)\cdots\Lambda(x_n)^{\Delta_n/2}\mathcal{O}_n(x_n)} \notag\\
    \ev{\mathcal{O}_1(x_1)\cdots\mathcal{O}_n(x_n)}
    &= \Lambda(x_1)^{-\Delta_1/2}\cdots\Lambda(x_n)^{-\Delta_n/2}\ev{\mathcal{O}_1(x'_1)\cdots\mathcal{O}_n(x'_n)}
\end{align}
というように変わることがわかる。%TODO: ほんとか?
これが、共形変換による相関関数の制限となる。
どこまで制限できるのか見ていく。

\subsection*{2点相関関数}
まずは2点相関関数を見ていく。
スケール変換\(x\rightarrow x'=\lambda x\)のもとでは\(\Lambda(x) =\lambda^{-2}\)より、
\begin{equation}
    \ev{\mathcal{O}_1(x_1){O}_2(x_2)}
    = \lambda^{\Delta_1 + \Delta_2}\ev{\mathcal{O}_1(x'_1)\mathcal{O}_2(x'_2)}
\end{equation}
これを満たす関数について考える。
並進変換と回転変換で変わらないことと、
2点相関関数でスカラーであることから、\(\ev{\mathcal{O}(x_1)\mathcal{O}(x_2)}\)は
\(\abs{x_1-x_2}\)の関数であることがわかる。
これより、\(f(x) = \ev{\mathcal{O}(x)\mathcal{O}(0)},\,\lambda=1+\epsilon\)とおいて、微分方程式を書くと
\begin{align*}
    f(x)
    &= (1+\epsilon)^{\Delta_1 + \Delta_2} f(x + \epsilon)\\
    &= f(x) + \epsilon\qty{(\Delta_1 + \Delta_2)f(x) + \dv{f}{x}\epsilon x}
\end{align*}
これを解くと
\begin{equation*}
    f(x) = \frac{C_{12}}{\abs{x}^{\Delta_1 + \Delta_2}}
\end{equation*}
となるのがわかる。
よって相関関数は
\begin{equation}
    \ev{\mathcal{O}(x_1)\mathcal{O}(x_2)}= \frac{C_{12}}{\abs{x_1-x_2}^{\Delta_1 + \Delta_2}}
\end{equation}

特殊共形変換を考える。
このとき計量は
\begin{equation*}
    g_{\mu\nu} \rightarrow g'_{\mu\nu} = (1+2B^\nu x_\nu +\abs{B}^2)^2 g_{\mu\nu}
\end{equation*}
のように変換するため、
\begin{equation*}
    \Lambda(x_j)^{-\Delta/2} = \frac{1}{(1+2B_\nu x_j^\nu +\abs{B}^2\abs{x}^2)^{\Delta}}
\end{equation*}
というようになる。
この分母を
\setcounter{equation}{67}
\begin{equation}
    \gamma_j = 1+2B_\nu x_j^\nu +\abs{B}^2\abs{x}^2
\end{equation}
というようにおくと、2点相関関数は
\setcounter{equation}{66}
\begin{equation}
    \ev{\mathcal{O}(x_1)\mathcal{O}(x_2)} = \frac{1}{\gamma_1^{\Delta_1}\gamma_2^{\Delta_2}}\ev{\mathcal{O}_1(x'_1)\mathcal{O}_2(x'_2)}
\end{equation}
と書ける。
\setcounter{equation}{68}
相関関数の形は(2.66)のような形をしているとわかっているので、
この形がどのように変換するかを考えればよい。
私は反転操作に慣れていないので思いつかないが、
このようにするとよい。
2点\(x'_1,\,x'_2\)は反転操作によって得られたものであるため、
このような量を考える。
\begin{equation*}
    \qty(\frac{s_1}{s_1}^2-\frac{s_2}{s_2^2})^2
    =\frac{(s_1-s_2)^2}{s_1^2s_2^2}
\end{equation*}
これより、
\begin{equation}
    (x'_1-x'_2)^2 = \dfrac{\qty{\qty(\dfrac{x_1}{x_1^2}+B)-\qty(\dfrac{x_2}{x_2^2}+B)}^2}{\qty(\dfrac{x_1}{x_1^2}+B)^2\qty(\dfrac{x_2}{x_2^2}+B)^2}
    =\dfrac{x_1^2x_2^2\qty(\dfrac{1}{x_1}-\dfrac{1}{x_2})^2}{\qty(1+Bx_1)^2\qty(1+Bx_2)^2} = \frac{(x_1-x_2)^2}{\gamma_1\gamma_2}
\end{equation}
となるので、(2.67)式の両辺は
\begin{equation}
    \frac{C_{12}}{\abs{x_1-x_2}}=\frac{(\gamma_1\gamma_2)^{(\Delta_1+\Delta_2)/2}}{\gamma_1^{\Delta_1}\gamma_2^{\Delta_2}}\frac{C_{12}}{\abs{x_1-x_2}}
\end{equation}
というように書ける。
これが任意の\(\gamma_1,\,\gamma_2\)で成り立つ必要があるので
\begin{equation*}
    \gamma_1^{(\Delta_2-\Delta_1)/2}\gamma_2^{(\Delta_1-\Delta_2)/2} = 1
\end{equation*}
つまり共形次元がおなじである\(\Delta_1=\Delta_2\)のときだけである。
よって2点相関関数は
\begin{equation}
    \ev{\mathcal{O}(x_1)\mathcal{O}(x_2)}= \frac{C_{12}\delta_{\Delta_1\Delta_2}}{\abs{x_1-x_2}^{\Delta_1 + \Delta_2}}
\end{equation}
という形まで制限される。
\footnote{自由スカラー場のときは対数だったのに対し、こっちはべき則になってる。なにかあるのかしら。}
ここの残った定数は準プライマリ場の再定義で吸収できるため物理的な意味のある量ではない。

\subsection*{3点相関関数}
次に3点の相関関数を考える。
2点と同様に並進変換と回転変換で変わらないこととから
この関数は
\begin{equation}
    r_{jl} = \abs{x_j-x_l}
\end{equation}
のみの関数であることがわかる。
また、スケール変換を考えると
\begin{align}
    \ev{\mathcal{O}'_1(x'_1)\mathcal{O}'_2(x'_2)\mathcal{O}'_3(x'_3)}
    &= \ev{\Lambda(x_1)^{\Delta_1/2}\mathcal{O}_1(x_1)\Lambda(x_2)^{\Delta_2/2}\mathcal{O}_n(x_2)\Lambda(x_3)^{\Delta_3/2}\mathcal{O}_3(x_3)} \notag\\
    \ev{\mathcal{O}_1(x_1)\mathcal{O}_2(x_2)\mathcal{O}_3(x_3)}
    &= \Lambda(x_1)^{-\Delta_1/2}\Lambda(x_2)^{-\Delta_2/2}\Lambda(x_3)^{-\Delta_3/2}\ev{\mathcal{O}_1(x'_1)\mathcal{O}_2(x'_2)\mathcal{O}_3(x'_3)}\notag\\
    &= \lambda^{\Delta_1+\Delta_2+\Delta_3}\ev{\mathcal{O}_1(x'_1)\mathcal{O}_2(x'_2)\mathcal{O}_3(x'_3)}\notag
\end{align}
である。
この式と(2.72)式を組み合わせる。
\begin{equation*}
    \ev{\mathcal{O}_1(x_1)\mathcal{O}_2(x_2)\mathcal{O}_3(x_3)} = f(r_{12},\,r_{23},\,r_{31})
\end{equation*}
として、スケール変換
\begin{equation*}
    (r'_{12},\,r'_{23},\,r'_{31})\rightarrow\lambda(r_{12},\,r_{23},\,r_{31}) = (1+\epsilon)(r_{12},\,r_{23},\,r_{31})
\end{equation*}
を入れると、
\begin{align*}
    f(r_{12},\,r_{23},\,r_{31})
        &= (1+\epsilon)^{\Delta_1+\Delta_2+\Delta_3}f(r'_{12},\,r'_{23},\,r'_{31})\\
        &= f(r_{12},\,r_{23},\,r_{31}) + \epsilon(\Delta_1+\Delta_2+\Delta_3)f(r_{12},\,r_{23},\,r_{31})\\
            &\qquad + \epsilon \qty(r_{12}\pdv{r_{12}}+r_{23}\pdv{r_{23}}+r_{31}\pdv{r_{31}})f(r_{12},\,r_{23},\,r_{31})\\
    -(\Delta_1+\Delta_2+\Delta_3)f(r_{12},\,r_{23},\,r_{31})
        &=\qty(r_{12}\pdv{r_{12}}+r_{23}\pdv{r_{23}}+r_{31}\pdv{r_{31}})f(r_{12},\,r_{23},\,r_{31})
\end{align*}
これを満たす\(f(r_{12},\,r_{23},\,r_{31})\)は
\begin{equation*}
    f(r_{12},\,r_{23},\,r_{31}) = \frac{C}{r_{12}^a r_{23}^b r_{31}^c},\quad a+b+c =\Delta_1+\Delta_2+\Delta_3
\end{equation*}
である。なので相関関数はこの形の線形結合
\begin{align}
    \ev{\mathcal{O}_1(x_1)\mathcal{O}_2(x_2)\mathcal{O}_3(x_3)}
        &= \sum_{a,\,b,\,c}\frac{C_{123}^{abc}}{r_{12}^a r_{23}^b r_{31}^c}\\
    a+b+c &=\Delta_1+\Delta_2+\Delta_3
\end{align}
がスケール変換により制限される形である。
さらに特殊共形変換に関してはさっきと同様にして
\begin{equation*}
    (x'_1-x'_2)^2(x'_2-x'_3)^2(x'_3-x'_1)^2 = \frac{(x_1-x_2)^2}{\gamma_1 \gamma_2}\frac{(x_2-x_3)^2}{\gamma_2 \gamma_3}\frac{(x_3-x_1)^2}{\gamma_3 \gamma_1}
\end{equation*}
という関係が示せるので
\begin{equation}
    \frac{C_{123}^{abc}}{r_{12}^a r_{23}^b r_{31}^c}
    = \frac{(\gamma_1\gamma_2)^{a/2}(\gamma_2\gamma_3)^{b/2}(\gamma_3\gamma_1)^{c/2}}{\gamma_1^{\Delta_1}\gamma_2^{\Delta_2}\gamma_3^{\Delta_3}}\frac{C_{123}^{abc}}{r_{12}^a r_{23}^b r_{31}^c}
\end{equation}
この等号が成り立つには
\begin{align}
    c+a = 2\Delta_1, \quad a+b &= 2\Delta_2, \quad b+c = 2\Delta_3 \\
    a &= \Delta_1 + \Delta_2 - \Delta_3\\
    b &= \Delta_1 - \Delta_2 + \Delta_3\\
    c &=-\Delta_1 + \Delta_2 + \Delta_3
\end{align}
というのが総和の添え字の条件となる。よって相関関数は
\begin{equation}
    \ev{\mathcal{O}_1(x_1)\mathcal{O}_2(x_2)\mathcal{O}_3(x_3)} = \frac{C_{123}}{r_{12}^{\Delta_1+\Delta_2-\Delta_3} r_{23}^{\Delta_1-\Delta_2+\Delta_3} r_{31}^{-\Delta_1+\Delta_2+\Delta_3}}\\
\end{equation}
という形になる。
なので座標依存性を一意的に決定できた。

\subsection*{4点相関関数}
では4点関数はどうなるか、
ここまでの式変形を追っていると、
並進変換と回転変換とスケール変換によって
\begin{equation}
    \ev{\mathcal{O}_1(x_1)\mathcal{O}_2(x_2)\mathcal{O}_3(x_3)\mathcal{O}_4(x_4)}
    = \sum_{a,\,b,\,c,\,d,\,e,\,f}\frac{C_{1234}^{abcdef}}{r_{12}^a r_{13}^b r_{14}^c r_{23}^d r_{24}^e r_{34}^f} \\
\end{equation}
という形になることがわかる。
ただ、特殊共形変換によって形は完全に決まるかというとそうではない。
実際(2.75)式に相当する式を作ると
\begin{equation*}
    \frac{C_{1234}^{abcdef}}{r_{12}^a r_{13}^b r_{14}^c r_{23}^d r_{24}^e r_{34}^f}
    =\frac{
        (\gamma_1\gamma_2)^{a/2}(\gamma_1\gamma_3)^{b/2}(\gamma_1\gamma_4)^{c/2}
        (\gamma_2\gamma_3)^{d/2}(\gamma_2\gamma_4)^{e/2}(\gamma_3\gamma_4)^{f/2}
    }{
        \gamma_1^{\Delta_1}\gamma_2^{\Delta_2}\gamma_3^{\Delta_3}\gamma_4^{\Delta_4}}
    \frac{C_{1234}^{abcdef}}{r_{12}^a r_{13}^b r_{14}^c r_{23}^d r_{24}^e r_{34}^f}
\end{equation*}
となり連立方程式を完全に解くことができず、不定性が残るのがわかる。
\footnote{おそらく、この不定性は対称群、交換子群、可解群、ガロア理論らへんと関係がある。}
この連立方程式を解くと
\begin{align*}
    a &= 2\Delta_1 - (b+c)\\
    d &= -\Delta_1 + \Delta_2 + \Delta_3 - \Delta_4 +c\\
    e &= -\Delta_1 + \Delta_2 - \Delta_3 + \Delta_4 +b\\
    f &= -\Delta_1 - \Delta_2 + \Delta_3 + \Delta_4 -(b+c)
\end{align*}
この自由度は連立方程式の Ker である。
よって Ker b部分は
\begin{equation*}
    a = -(b+c),\quad d = c,\quad e = b ,\quad f = -(b+c)
\end{equation*}
なので相関関数には
\begin{equation}
    \frac{r_{12}r_{34}}{r_{13}r_{24}},\quad \frac{r_{12}r_{34}}{r_{14}r_{23}}
\end{equation}
というのを自由に掛けることができるので、
\begin{equation*}
    \ev{\mathcal{O}_1(x_1)\mathcal{O}_2(x_2)\mathcal{O}_3(x_3)\mathcal{O}_4(x_4)}
    \propto F\qty(\frac{r_{12}r_{34}}{r_{13}r_{24}},\, \frac{r_{12}r_{34}}{r_{14}r_{23}})
\end{equation*}
という形となる。

Ker の分の自由度はどのようにとっても良いが、
\begin{align*}
    b &= \frac{1}{3}(2\Delta_1 - \Delta_2 +2\Delta_3 - \Delta_4)\\
    c &= \frac{1}{3}(2\Delta_1 - \Delta_2 - \Delta_3 +2\Delta_4)
\end{align*}
というようにすると、総和の添え字は\(\Delta = \Delta_1 + \Delta_2 + \Delta_3 + \Delta_4\)として
\begin{align*}
    a &= \frac{1}{3}(2\Delta_1 +2\Delta_2 - \Delta_3 - \Delta_4) = \Delta_1 -\Delta_2 -\frac{\Delta}{3}\\
    b &= \frac{1}{3}(2\Delta_1 - \Delta_2 +2\Delta_3 - \Delta_4) = \Delta_1 -\Delta_3 -\frac{\Delta}{3}\\
    c &= \frac{1}{3}(2\Delta_1 - \Delta_2 - \Delta_3 +2\Delta_4) = \Delta_1 -\Delta_4 -\frac{\Delta}{3}\\
    d &= \frac{1}{3}( \Delta_1 +2\Delta_2 +2\Delta_3 - \Delta_4) = \Delta_2 -\Delta_3 -\frac{\Delta}{3}\\
    e &= \frac{1}{3}( \Delta_1 +2\Delta_2 - \Delta_3 +2\Delta_4) = \Delta_2 -\Delta_4 -\frac{\Delta}{3}\\
    f &= \frac{1}{3}( \Delta_1 - \Delta_2 +2\Delta_3 +3\Delta_4) = \Delta_3 -\Delta_4 -\frac{\Delta}{3}
\end{align*}
というようにきれいな形になる。
これを使うと4点相関関数は
\begin{equation}
    \ev{\mathcal{O}_1(x_1)\mathcal{O}_2(x_2)\mathcal{O}_3(x_3)\mathcal{O}_4(x_4)}
    =  F\qty(\frac{r_{12}r_{34}}{r_{13}r_{24}},\, \frac{r_{12}r_{34}}{r_{14}r_{23}})
    \prod_{j>l} r_{jl}^{\Delta/3-(\Delta_j+\Delta_l)}
\end{equation}
という形となる。
ここで出てきた Ker の部分
\begin{equation}
    \frac{r_{ij}r_{kl}}{r_{ik}r_{jl}}
\end{equation}
は交差比と呼ばれている。

\section{エネルギー運動量テンソル}
並進運動に対応する保存カレントであるエネルギー運動量テンソルは
共形場理論において非常に重要な役割を果たすらしい。
変換パラメータを\(x^\mu\rightarrow x'^\mu = x^\mu -\epsilon^\mu(x)\)というように位置に依存するとき、
保存カレントと作用の変分の関係は(1.81)式より、
\begin{equation}
    \var{S} = \int d^dx\,T^{\mu\nu}\partial_\mu \epsilon_\nu(x)
\end{equation}
となる。
一方、カレントには
\begin{equation}
    j^\mu \rightarrow = j'^\mu + \partial_\nu B^{\nu\mu}, \quad B^{\nu\mu} = -B^{\mu\nu}
\end{equation}
のように反対称テンソルの微分の分だけの不定性がある。
この不定性によりエネルギー運動量テンソルを対称テンソルにすることができる。
\begin{equation*}
    T'^{\mu\nu} = T^{\mu\nu} + \partial_\rho B^{\rho\mu\nu},\quad B^{\rho\mu\nu} =- B^{\mu\rho\nu}
\end{equation*}
このように対称化したエネルギー運動量テンソルを Belinfante エネルギー運動量テンソルと呼ばれる。
対称にしたとき、
作用の変分は
\begin{equation}
    \var{S} = \frac{1}{2}\int d^dx\,T^{\mu\nu} (\partial_\mu \epsilon_\nu + \partial_\nu \epsilon_\mu)
\end{equation}
というようになる。

この形は共形変換における計量の変化に似ているのでこれと比較してみよう。
共形変換のときの計量の変化とは微妙に違うが、
\(\var{g}_{\mu\nu}=\partial_\mu\epsilon_\nu+\partial_\nu\epsilon_\mu\)というように変えてみる。
これこの節で初めにした変換に対応する計量の変化である。
これを使うと作用の変分は
\begin{equation}
    \var{S} = \int d^dx\, \fdv{S}{g_{\mu\nu}}\,(\partial_\mu\epsilon_\nu+\partial_\nu\epsilon_\mu)
\end{equation}
となる。
作用の力学変数を座標と計量の2つがあるとすると\footnote{いままでは平坦なユークリッド計量なので計量の変分は考えてなかった。ここでしか曲がった時空は考えたない。}、
作用の変分の合計は
\begin{equation*}
    \var{S} = \int d^dx\,\qty(\frac{1}{2}T^{\mu\nu} + \fdv{S}{g_{\mu\nu}})(\partial_\mu\epsilon_\nu+\partial_\nu\epsilon_\mu)
\end{equation*}
となる。これが 0 になるのでエネルギー運動量テンソルは
\begin{equation}
    T^{\mu\nu} = -2 \fdv{S}{g_{\mu\nu}}
\end{equation}
となる。
一般に曲がった背景上では
\begin{equation}
    T^{\mu\nu} = -\frac{2}{\sqrt{\det g}}\fdv{S}{g_{\mu\nu}}
\end{equation}
である。
\footnote{重力との関連? 一般相対論の結果をさらっと引用? 曲がった背景上での場の量子論とのつながり?}

では共形変換をしてみよう。
この場合、計量は\(\var{g_{\mu\nu}}=\partial_\mu\epsilon_\nu+\partial_\nu\epsilon_\mu=\frac{2}{d}\partial_\rho \epsilon^\rho g_{\mu\nu}\)と変化する。
なのでこれを(2.87)にいれて、
\begin{equation}
    \var{S}
    = \frac{1}{d} \int d^dx\, T^{\mu\nu}\frac{2}{d}\partial_\rho \epsilon^\rho g_{\mu\nu}
    = \frac{1}{d}\int d^dx\, T\indices{^\mu_\mu}\partial_\nu\epsilon^\nu
\end{equation}
となる。
任意の共形変換に対して作用の変分が 0 となるのは
\begin{equation}
    T\indices{^\mu_\mu} =0
\end{equation}
というトレースレス条件が付く\footnote{これ結局どういう状況?}。

さらに具体的な変換をしてみる。
回転変換は
\begin{equation}
    x^\mu\rightarrow x'^\mu = x^\mu -\epsilon^\mu,\quad \epsilon^\mu = \epsilon^{\mu\nu}x_\nu, \quad(\epsilon^{\mu\nu}= - \epsilon^{\nu\mu})
\end{equation}
というように表される。
このとき、スカラー場とラグランジアンは
\begin{align}
    \phi(x)&\rightarrow \phi'(x) = \phi(x) + \epsilon^{\mu\nu}x_\nu\partial_\mu\phi(x)\\
    \mathcal{L}(x)&\rightarrow \mathcal{L}'(x)
    = \mathcal{L}(x) + \epsilon^{\mu\nu}x_\nu\partial_\mu\mathcal{L}(x)
    = \mathcal{L}(x) + \epsilon^{\mu\nu}x_\nu\partial_\mu\mathcal{L}(x)
\end{align}
というように変換する。
作用の変分は
\begin{align*}
    \var{S}
    &= \int d^dx\, \qty{\pdv{\mathcal{L}}{(\partial_\mu\phi)}\partial_\nu\phi - \delta_\nu^\mu \mathcal{L}}(\partial_\mu \epsilon^\nu)\\
    &= \int d^dx\, \qty{\pdv{\mathcal{L}}{(\partial_\mu\phi)}\partial_\nu\phi - \delta_\nu^\mu \mathcal{L}}(\partial_\mu \epsilon^{\nu\rho}x_\rho)\\
\end{align*}


なのでカレントは
\begin{align}
    j\indices{^\mu_\nu_\rho} = \pdv{\mathcal{L}}{(\partial_\mu \phi)} L_{\nu\rho}\phi
\end{align}


\end{document}