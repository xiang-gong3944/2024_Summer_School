\documentclass[11pt,dvipdfmx,a4paper]{jsreport}

\usepackage[margin=15mm]{geometry} %余白の削除

\usepackage{subfiles}

\usepackage{amsmath,amssymb}
\usepackage{bm}
\usepackage[dvipdfmx]{graphicx}
\usepackage{physics} % http://mirrors.ibiblio.org/CTAN/macros/latex/contrib/physics/physics.pdf
\usepackage{siunitx} %SI単位を楽に出力
\usepackage{mathtools} %環境の追加
\usepackage{pdfpages}
\usepackage[subrefformat=parens]{subcaption} %(a)図のようにすることができるやつ
\usepackage{here}
\usepackage{mathrsfs}
\usepackage{url}
% \usepackage{circuitikz} %電気回路をtex中で書く
% \usepackage{caption} %番号なしキャプションを書く
\usepackage{cancel} %式中に斜線を入れる
\usepackage{tensor} %テンソルの添え字を書く
% \usepackage{tikz} %図を書く
% \usepackage{ascmac} %四角い枠の中に文章を書く
% \usepackage{float} %figureで[hbp]オプションを使う
% \usepackage{hyperref}  \usepackage{pxjahyper} %ハイパーリンクをつかう
% \usepackage{tablefootnote} %表中に注釈をいれる
% \usepackage[thicklines]{cancel} %数式中の取り消し線
% \usepackage[version=4]{mhchem} %化学式の入力
% \usepackage{wrapfig} %文章の回り込み

% \graphicspath{{./image/}}

% \numberwithin{equation}{section} %式番号を(セクション.式番号)にする

\newcommand{\underlineset}[2]{\underset{#1}{\underline{#2}}}

\title{共形場理論入門 基礎からホログラフィへの道 : ノート}
\author{しゃんごん (@CEY3944)}
\date{\today}

\begin{document}

\maketitle
\tableofcontents

% コメントアウトにより master に入れるか入れないかを決める。
\clearpage
\subfile{./content/01/chapter.tex}
\clearpage
\subfile{./content/02/chapter.tex}
\clearpage
\subfile{./content/03/chapter.tex}
% \clearpage
% \subfile{./content/04/chapter.tex}
% \clearpage
% \subfile{./content/05/chapter.tex}
% \clearpage
% \subfile{./content/06/chapter.tex}
% \clearpage
% \subfile{./content/07/chapter.tex}
% \clearpage
% \subfile{./content/08/chapter.tex}
% \clearpage
% \subfile{./content/09/chapter.tex}


\bibliographystyle{junsrt}
\bibliography{reference}
\nocite{*}

\end{document}